\section{Parametric filtering using the Z-transform}
\subsection{The Z-transform of a signal}
For a real time domain signal $x:\mathbb{Z}\to\mathbb{R}$, its $z$ transform is a function $X:\mathbb{C}\to\mathbb{C}$, where
$$X(z)=\sum_{n=-\infty}^{\infty}x[n]z^{-n}.$$

It can be seen that the DTFT is the same as the Z-transform but with the domain restricted to the 
unit circle on the complex plane.

\subsection{The transfer function of a system}
For an LTI system suppose that $x$ is an arbitrary input signal and $y$ is the output.
Let $X$ and $Y$ be the Z-transforms of these signals respectively.

$$H(z)=\frac{Y(z)}{X(z)}$$

The transfer function is not dependent on which arbitrary input signal is chosen.

\subsection{Properties of Z-transforms}
\textbf{Linearity}\\
Let $a$ and $b$ be arbitrary signals. Let $A$ and $B$ be their Z-transforms respectively.
The Z-transform of $a+b$ is equal to $A(z) + B(z)$.

\textbf{Shift Property}\\
Let $a$ be an arbitrary signal and define $b$ such that $b[t] = a[t-n]$ (time shift).
Then the Z-transform of $b$ is $A(z)z^{-n}$.

\subsection{Recurrence formula to transfer function}
Suppose 

\subsection{Transfer function to recurrence formula}

\subsection{Desmos visualization of the transfer function}
I created a demonstration on how the positions of poles and zeroes in the 
transfer function of the system impact the frequency magnitude response of the system.
\href{https://www.desmos.com/calculator/cl4nz0pdtd}{https://www.desmos.com/calculator/cl4nz0pdtd}

\subsection{Biquadratic filter implementation}
This is implemented recursively.