\section{Implementing a Brick-Wall Filter}

The link to the GitHub repo is \href{https://github.com/mt-ra/Audio-Processing-Experiment}{https://github.com/mt-ra/Audio-Processing-Experiment}.

The goal was to create a program that processes a .wav file. 
This program should have the ability to apply a brickwall filter, 
either high pass or low pass.

Initially I had issues with clicking being heard in the output, 
before I learned that I was supposed to do \emph{windowing} on the signal before
applying the DFT.

The dependencies of the program are:
\begin{itemize}
    \item Boost/circular\_buffer
    \item Portsf (C library for reading and writing wav files)
    \item FFTW (C library for FFT)
\end{itemize}

\subsection{Writing wrappers for FFTW}
I started by writing C++ OOP style wrappers around the FFTW library. 
I defined a FFT \emph{transformer} class with methods to perform either a DFT or an IDFT on a vector of complex numbers.
There needed to be object state because FFTW uses \emph{FFTW plans} which are expensive to compute, but
can be used to perform multiple transformations on the same buffer, throughout the lifetime of the object.

Furthermore, RAII principles could be used to ensure that resources allocated for the buffer and the FFTW 
plan are freed when the transformer object goes out of scope.
The copy constructor and copy assignment operators were also deleted for similar reasons,
to ensure the validity of resource management.

Below is the class definition of the Transformer class, as it is written in the header file \verb|Transformer.h|.
\begin{verbatim}
class Transformer {
    size_t n;
    fftw_complex *in_buf;
    fftw_complex *out_buf;
    fftw_plan forward_plan;
    fftw_plan backward_plan;

public:
    Transformer(size_t n_);
    ~Transformer();
    Transformer(Transformer& t) = delete;
    Transformer& operator=(const Transformer& other) = delete;

public:
    std::vector<std::complex<double>> dft(const std::vector<std::complex<double>>& c) const;
    std::vector<std::complex<double>> idft(const std::vector<std::complex<double>>& ch) const;
};
\end{verbatim}

\pagebreak

\subsection{Writing wrappers for Portsf}
My only use for Portsf was for reading frames of a wav file into my program,
and writing frames to an output file. 
I tried to write a class similar to C++'s standard input stream and output stream,
where the bitshift operators are overloaded to perform stream extraction / insertion.
Writing a class wrapper for Portsf was also important for RAII, as there
were also resources associated with the files that were open for reading and writing
that I needed to manage.

The goal was to be able to use my classes in more or less the following way:
\begin{verbatim}
{
    std::vector<std::complex<double>> inbuf(1024, 0);
    psfin >> inbuf;
    process(inbuf);
    psfout << inbuf;
}
\end{verbatim}

Below are the class definitions for my custom istream and ostream classes:
\begin{verbatim}
class psf_istream {
    int in_fd;
    int size;
    int position;
    PSF_PROPS props;

public:
    psf_istream(std::string filename);
    ~psf_istream();

public:
    friend psf_istream& operator>>(psf_istream& in, std::vector<std::complex<double>>& buf);

    PSF_PROPS get_props();

    bool eof();
};

class psf_ostream {
    int out_fd;
    PSF_PROPS props;

public:
    psf_ostream(std::string filename, PSF_PROPS props_);
    ~psf_ostream();

public:
    friend psf_ostream& operator<<(psf_ostream& out, std::vector<std::complex<double>> buf);

    PSF_PROPS get_props();
};
\end{verbatim}


\subsection{Program architecture}
I wanted to be able to read from the portsf input stream into a vector.
And then write this vector into an object which processes the samples and writes to the portsf output stream.

\subsection{The StreamProcessor Class}
A StreamProcessor object is fed audio signal samples, and when sufficient samples
are stored into the pre-buffer, it builds windows of samples to process.

I originally intended for the StreamProcessor class to be templated so that 
it could send its output to any client specified output stream.
This was so that the client could decide to write samples to the input buffer
at any rate, and to extract the processed samples from the output buffer
at any rate less than this.
However I decided this would be too much work and just made it so that you 
needed pass a \verb|psf_ostream| object by reference into the constructor, so that
the contents of the output buffer were automatically writen to a \verb|psf_ostream|.

Samples are written to the StreamProcessor object via the left shift operator.
Overlapping windows are constructed and processed using a choice of functions,
such as the brick wall filters or FIR filters.

Below is the class definition
\begin{verbatim}
class StreamProcessor {
    boost::circular_buffer<std::complex<double>> pre_buf;
    boost::circular_buffer<std::complex<double>> post_buf;
    size_t window_size;
    size_t sample_rate;
    psf_ostream& psfout;

    std::vector<std::complex<double>> hann;
    Transformer transformer;

public:
    StreamProcessor(   
        psf_ostream& psfout_, 
        size_t window_size_
    );

public:
    friend StreamProcessor& operator<<(
        StreamProcessor& proc, 
        std::vector<std::complex<double>> in
    );

private:
    void try_to_consume_pre_buf();
    void try_to_consume_post_buf();
};
\end{verbatim}

\subsection{Brickwall Filter Functions}
The \verb|brickwall_lowpass| function has the following prototype:

\begin{verbatim}
std::vector<std::complex<double>> brickwall_lowpass(
    std::vector<std::complex<double>> const& signal,
    size_t cutoff,
    Transformer const& t
);
\end{verbatim}

Quite simply, the function takes a windowed set of samples, and performs a DFT.
In the frequency domain, all the components associated with frequencies above 
the cutoff frequency are set to zero.
An IDFT is performed on the result.


\subsection{FIR Filtering}
To be discussed in the next section.